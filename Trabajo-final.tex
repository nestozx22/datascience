% Options for packages loaded elsewhere
\PassOptionsToPackage{unicode}{hyperref}
\PassOptionsToPackage{hyphens}{url}
%
\documentclass[
]{article}
\title{Trabajo final Data Science - Grupo 1}
\author{}
\date{\vspace{-2.5em}}

\usepackage{amsmath,amssymb}
\usepackage{lmodern}
\usepackage{iftex}
\ifPDFTeX
  \usepackage[T1]{fontenc}
  \usepackage[utf8]{inputenc}
  \usepackage{textcomp} % provide euro and other symbols
\else % if luatex or xetex
  \usepackage{unicode-math}
  \defaultfontfeatures{Scale=MatchLowercase}
  \defaultfontfeatures[\rmfamily]{Ligatures=TeX,Scale=1}
\fi
% Use upquote if available, for straight quotes in verbatim environments
\IfFileExists{upquote.sty}{\usepackage{upquote}}{}
\IfFileExists{microtype.sty}{% use microtype if available
  \usepackage[]{microtype}
  \UseMicrotypeSet[protrusion]{basicmath} % disable protrusion for tt fonts
}{}
\makeatletter
\@ifundefined{KOMAClassName}{% if non-KOMA class
  \IfFileExists{parskip.sty}{%
    \usepackage{parskip}
  }{% else
    \setlength{\parindent}{0pt}
    \setlength{\parskip}{6pt plus 2pt minus 1pt}}
}{% if KOMA class
  \KOMAoptions{parskip=half}}
\makeatother
\usepackage{xcolor}
\IfFileExists{xurl.sty}{\usepackage{xurl}}{} % add URL line breaks if available
\IfFileExists{bookmark.sty}{\usepackage{bookmark}}{\usepackage{hyperref}}
\hypersetup{
  pdftitle={Trabajo final Data Science - Grupo 1},
  hidelinks,
  pdfcreator={LaTeX via pandoc}}
\urlstyle{same} % disable monospaced font for URLs
\usepackage[margin=1in]{geometry}
\usepackage{color}
\usepackage{fancyvrb}
\newcommand{\VerbBar}{|}
\newcommand{\VERB}{\Verb[commandchars=\\\{\}]}
\DefineVerbatimEnvironment{Highlighting}{Verbatim}{commandchars=\\\{\}}
% Add ',fontsize=\small' for more characters per line
\usepackage{framed}
\definecolor{shadecolor}{RGB}{248,248,248}
\newenvironment{Shaded}{\begin{snugshade}}{\end{snugshade}}
\newcommand{\AlertTok}[1]{\textcolor[rgb]{0.94,0.16,0.16}{#1}}
\newcommand{\AnnotationTok}[1]{\textcolor[rgb]{0.56,0.35,0.01}{\textbf{\textit{#1}}}}
\newcommand{\AttributeTok}[1]{\textcolor[rgb]{0.77,0.63,0.00}{#1}}
\newcommand{\BaseNTok}[1]{\textcolor[rgb]{0.00,0.00,0.81}{#1}}
\newcommand{\BuiltInTok}[1]{#1}
\newcommand{\CharTok}[1]{\textcolor[rgb]{0.31,0.60,0.02}{#1}}
\newcommand{\CommentTok}[1]{\textcolor[rgb]{0.56,0.35,0.01}{\textit{#1}}}
\newcommand{\CommentVarTok}[1]{\textcolor[rgb]{0.56,0.35,0.01}{\textbf{\textit{#1}}}}
\newcommand{\ConstantTok}[1]{\textcolor[rgb]{0.00,0.00,0.00}{#1}}
\newcommand{\ControlFlowTok}[1]{\textcolor[rgb]{0.13,0.29,0.53}{\textbf{#1}}}
\newcommand{\DataTypeTok}[1]{\textcolor[rgb]{0.13,0.29,0.53}{#1}}
\newcommand{\DecValTok}[1]{\textcolor[rgb]{0.00,0.00,0.81}{#1}}
\newcommand{\DocumentationTok}[1]{\textcolor[rgb]{0.56,0.35,0.01}{\textbf{\textit{#1}}}}
\newcommand{\ErrorTok}[1]{\textcolor[rgb]{0.64,0.00,0.00}{\textbf{#1}}}
\newcommand{\ExtensionTok}[1]{#1}
\newcommand{\FloatTok}[1]{\textcolor[rgb]{0.00,0.00,0.81}{#1}}
\newcommand{\FunctionTok}[1]{\textcolor[rgb]{0.00,0.00,0.00}{#1}}
\newcommand{\ImportTok}[1]{#1}
\newcommand{\InformationTok}[1]{\textcolor[rgb]{0.56,0.35,0.01}{\textbf{\textit{#1}}}}
\newcommand{\KeywordTok}[1]{\textcolor[rgb]{0.13,0.29,0.53}{\textbf{#1}}}
\newcommand{\NormalTok}[1]{#1}
\newcommand{\OperatorTok}[1]{\textcolor[rgb]{0.81,0.36,0.00}{\textbf{#1}}}
\newcommand{\OtherTok}[1]{\textcolor[rgb]{0.56,0.35,0.01}{#1}}
\newcommand{\PreprocessorTok}[1]{\textcolor[rgb]{0.56,0.35,0.01}{\textit{#1}}}
\newcommand{\RegionMarkerTok}[1]{#1}
\newcommand{\SpecialCharTok}[1]{\textcolor[rgb]{0.00,0.00,0.00}{#1}}
\newcommand{\SpecialStringTok}[1]{\textcolor[rgb]{0.31,0.60,0.02}{#1}}
\newcommand{\StringTok}[1]{\textcolor[rgb]{0.31,0.60,0.02}{#1}}
\newcommand{\VariableTok}[1]{\textcolor[rgb]{0.00,0.00,0.00}{#1}}
\newcommand{\VerbatimStringTok}[1]{\textcolor[rgb]{0.31,0.60,0.02}{#1}}
\newcommand{\WarningTok}[1]{\textcolor[rgb]{0.56,0.35,0.01}{\textbf{\textit{#1}}}}
\usepackage{longtable,booktabs,array}
\usepackage{calc} % for calculating minipage widths
% Correct order of tables after \paragraph or \subparagraph
\usepackage{etoolbox}
\makeatletter
\patchcmd\longtable{\par}{\if@noskipsec\mbox{}\fi\par}{}{}
\makeatother
% Allow footnotes in longtable head/foot
\IfFileExists{footnotehyper.sty}{\usepackage{footnotehyper}}{\usepackage{footnote}}
\makesavenoteenv{longtable}
\usepackage{graphicx}
\makeatletter
\def\maxwidth{\ifdim\Gin@nat@width>\linewidth\linewidth\else\Gin@nat@width\fi}
\def\maxheight{\ifdim\Gin@nat@height>\textheight\textheight\else\Gin@nat@height\fi}
\makeatother
% Scale images if necessary, so that they will not overflow the page
% margins by default, and it is still possible to overwrite the defaults
% using explicit options in \includegraphics[width, height, ...]{}
\setkeys{Gin}{width=\maxwidth,height=\maxheight,keepaspectratio}
% Set default figure placement to htbp
\makeatletter
\def\fps@figure{htbp}
\makeatother
\setlength{\emergencystretch}{3em} % prevent overfull lines
\providecommand{\tightlist}{%
  \setlength{\itemsep}{0pt}\setlength{\parskip}{0pt}}
\setcounter{secnumdepth}{-\maxdimen} % remove section numbering
\ifLuaTeX
  \usepackage{selnolig}  % disable illegal ligatures
\fi

\begin{document}
\maketitle

\hypertarget{brechas-de-ciberseguridad---estados-unidos}{%
\section{Brechas de ciberseguridad - Estados
Unidos}\label{brechas-de-ciberseguridad---estados-unidos}}

-Ernesto Guarda\\
-Christian Vadillo\\
-Cristhian Medina\\
-Jorge Cabrera\\
-Royer Rojas

\hypertarget{objetivo}{%
\subsection{1 Objetivo}\label{objetivo}}

El objetivo de esta práctica es realizar un estudio de las brechas de
ciberseguridad que se han dado en Estados Unidos entre los años 2000 y
2014.

Luego de presentar la estructura y los trabajos realizados en el dataset
se procederá a responder las preguntas que se han planteado con el fin
de obtener información para orientarnos a tomar decisiones.

\hypertarget{libreruxedas-utilizadas}{%
\subsubsection{1.1 Librerías utilizadas}\label{libreruxedas-utilizadas}}

Para poder elaborar este script hemos utilizado las siguientes librerias
de R:

\begin{enumerate}
\def\labelenumi{\arabic{enumi}.}
\tightlist
\item
  readr
\item
  dplyr
\item
  ggplot2
\item
  tidyverse
\item
  ggthemes
\item
  lubridate
\item
  lattice
\item
  survival
\item
  Formula
\item
  Hmisc
\end{enumerate}

\hypertarget{dataset}{%
\subsubsection{1.2 Dataset}\label{dataset}}

El dataset utilizado se llama ``Cyber Security Breaches'' y puede se
encontrado dando click
\href{https://www.kaggle.com/alukosayoenoch/cyber-security-breaches-data}{aquí}

\hypertarget{preparaciuxf3n-del-dataset}{%
\subsubsection{1.3 Preparación del
dataset}\label{preparaciuxf3n-del-dataset}}

Antes de comenzar a trabajar con el dataset ajustaremos los tipos de
variables para poder obtener resultados correctos

\begin{Shaded}
\begin{Highlighting}[]
\CommentTok{\#La variable "State" la convertiremos a tipo factor}
\NormalTok{cyberb}\SpecialCharTok{$}\NormalTok{State }\OtherTok{\textless{}{-}} \FunctionTok{as.factor}\NormalTok{(cyberb}\SpecialCharTok{$}\NormalTok{State)}
\CommentTok{\#La variable "Type\_of\_Breach" la convertiremos a tipo factor}
\NormalTok{cyberb}\SpecialCharTok{$}\NormalTok{Type\_of\_Breach }\OtherTok{\textless{}{-}} \FunctionTok{as.factor}\NormalTok{(cyberb}\SpecialCharTok{$}\NormalTok{Type\_of\_Breach)}
\CommentTok{\#La variable "Location\_of\_Breached\_Information" la convertiremos a tipo factor}
\NormalTok{cyberb}\SpecialCharTok{$}\NormalTok{Location\_of\_Breached\_Information }\OtherTok{\textless{}{-}} \FunctionTok{as.factor}\NormalTok{(cyberb}\SpecialCharTok{$}\NormalTok{Location\_of\_Breached\_Information)}
\CommentTok{\#La variable "Date\_Posted\_or\_Updated" la convertiremos a tipo fecha}
\NormalTok{cyberb}\SpecialCharTok{$}\NormalTok{Date\_Posted\_or\_Updated }\OtherTok{\textless{}{-}} \FunctionTok{as.Date}\NormalTok{(cyberb}\SpecialCharTok{$}\NormalTok{Date\_Posted\_or\_Updated,}\AttributeTok{format=}\StringTok{"\%d/\%m/\%Y"}\NormalTok{)}
\CommentTok{\#La variable "breach\_start" la convertiremos a tipo fecha}
\NormalTok{cyberb}\SpecialCharTok{$}\NormalTok{breach\_start }\OtherTok{\textless{}{-}} \FunctionTok{as.Date}\NormalTok{(cyberb}\SpecialCharTok{$}\NormalTok{breach\_start,}\AttributeTok{format=}\StringTok{"\%d/\%m/\%Y"}\NormalTok{)}
\CommentTok{\#La variable "breach\_end" la convertiremos a tipo fecha}
\NormalTok{cyberb}\SpecialCharTok{$}\NormalTok{breach\_end }\OtherTok{\textless{}{-}} \FunctionTok{as.Date}\NormalTok{(cyberb}\SpecialCharTok{$}\NormalTok{breach\_end,}\AttributeTok{format=}\StringTok{"\%d/\%m/\%Y"}\NormalTok{)}
\end{Highlighting}
\end{Shaded}

\hypertarget{descripciuxf3n-de-las-variables-del-dataset}{%
\subsection{1.4 Descripción de las variables del
dataset}\label{descripciuxf3n-de-las-variables-del-dataset}}

\begin{verbatim}
## cyberb 
## 
##  10  Variables      1055  Observations
## --------------------------------------------------------------------------------
## Name_of_Covered_Entity 
##        n  missing distinct 
##     1055        0      963 
## 
## lowest : 101 FAMILY MEDICAL GROUP                                                                     ABQ HealthPartners                                                                           Abrham Tekola, M.D.,INC                                                                      Accendo                                                                                      AccentCare Home Health of California, Inc. Medicare # 057564    CA state License # 080000226
## highest: Yale University                                                                              Yanez Dental Corporation                                                                     Yellowstone Boys and Girls Ranch                                                             Young Family Medicine Inc.                                                                   zarzamora family dental care                                                                
## --------------------------------------------------------------------------------
## State 
##        n  missing distinct 
##     1055        0       52 
## 
## lowest : AK AL AR AZ CA, highest: VT WA WI WV WY
## --------------------------------------------------------------------------------
## Business_Associate_Involved 
##        n  missing distinct 
##      271      784      214 
## 
## lowest : Accretive Health                                         Accuprint                                                ACS, Affiliated Computer Services, Inc., A Xerox Company ADPI-West                                                Advanced Data Processing Inc                            
## highest: Xand Corporation                                         Xforia Web Services                                      Yadkinville Chiropractic DCPA                            ZDI                                                      Zenith Administrators, Inc.                             
## --------------------------------------------------------------------------------
## Individuals_Affected 
##        n  missing distinct     Info     Mean      Gmd      .05      .10 
##     1055        0      809        1    30262    55209      550      629 
##      .25      .50      .75      .90      .95 
##     1000     2300     6941    20446    55062 
## 
## lowest :     500     501     502     504     505
## highest: 1220000 1700000 1900000 4029530 4900000
## --------------------------------------------------------------------------------
## Type_of_Breach 
##        n  missing distinct 
##     1055        0       28 
## 
## lowest : Hacking/IT Incident                                        Hacking/IT Incident, Other                                 Improper Disposal                                          Improper Disposal, Unauthorized Access/Disclosure          Loss                                                      
## highest: Unauthorized Access/Disclosure, Hacking/IT Incident        Unauthorized Access/Disclosure, Hacking/IT Incident, Other Unauthorized Access/Disclosure, Other                      Unknown                                                    Unknown, Other                                            
## --------------------------------------------------------------------------------
## Location_of_Breached_Information 
##        n  missing distinct 
##     1055        0       41 
## 
## lowest : Desktop Computer                                                           Desktop Computer, E-mail                                                   Desktop Computer, Electronic Medical Record                                Desktop Computer, Network Server                                           Desktop Computer, Network Server, E-mail, Electronic Medical Record, Paper
## highest: Other Portable Electronic Device, Other                                    Other Portable Electronic Device, Other, Electronic Medical Record         Other, Electronic Medical Record                                           Other, Paper                                                               Paper                                                                     
## --------------------------------------------------------------------------------
## Date_Posted_or_Updated 
##          n    missing   distinct       Info       Mean        Gmd        .05 
##       1055          0         43      0.719 2014-02-23      47.55 2014-01-23 
##        .10        .25        .50        .75        .90        .95 
## 2014-01-23 2014-01-23 2014-01-23 2014-03-24 2014-06-03 2014-06-19 
## 
## lowest : 2014-01-23 2014-01-24 2014-01-31 2014-02-11 2014-02-12
## highest: 2014-06-19 2014-06-20 2014-06-24 2014-06-27 2014-06-30
## --------------------------------------------------------------------------------
## Summary 
##        n  missing distinct 
##      142      913      141 
## 
## lowest : 
## 
## OCR opened an investigation of the covered entity (CE), Paul G. Klein DPM, after it reported an encrypted and password protected laptop was stolen that contained the electronic protected health information (ePHI) of 2,500 individuals.  The ePHI included names, addresses, dates of birth, social security numbers, diagnosis conditions, lab test results, medications, medical notes, and treatment plans.  Upon discovery of the breach, the CE filed a police report to recover the stolen item.  As a result of OCR's investigation, the CE provided confirmation that there was encryption software and multi-layered password protection software installed on the stolen laptop.  OCR determined that the impermissible disclosure of ePHI did not constitute a breach under the Privacy Rule's breach notification rule and provided technical assistance to the CE regarding the requirements of the breach notification rule.
## 
## 
## 
##                                                                                                                                                                                                                                           
## 
## The covered entity (CE), Medco Health Solutions, mailed letters with incorrect addresses after a programming code in its mailing software caused corruption of its data.  The mailing contained the protected health information (PHI) of 4,341 individuals and included names, medication name and prescription number.  The CE provided breach notification to HHS, the media, and affected individuals.  Upon discovery of the breach, the CE immediately ceased using the update to its mailing software system.  As a result of OCR's investigation, the CE corrected the update to its mailing software system and established manual and automated quality control processes.  The breach incident involved a BA and occurred prior to the September 23, 2013, compliance date.  OCR verified that the CE had a proper BA agreement in place that restricted the BA's use and disclosure of PHI and required the BA to safeguard all PHI.
## 
##                                                                                                                                                                                                                                            22 computers were stolen from Clinical Management Service office.Five of the stolen computers contained the protected health information of approximately 22,012 individuals. The protected health information involved in the breach included name, date of birth, social security number, referral number, encounter number, facility, member ID, diagnosis, procedure, and/or diagnosis code. As a result of this incident, St. Joseph notified the potentially affected individuals, notified the local media, installed security cameras, re-trained employees, and installed encryption software on all laptops and Computers enterprise-wide. OCR's investigation resulted in the covered entity improving their physical and technological safeguards and retraining employees.
## 
##                                                                                                                                                                                                                                                                                                                                                                                                         A bag containing 43 pages of protected health information (PHI) of 550 nursing home residents and an encrypted laptop computer were stolen from the vehicle of an employee of the covered entity's (CE) business associate (BA).  The PHI included names, dates of birth, gender identities, names of the nursing homes, and Medicaid numbers.  Upon discovery of the breach, the CE filed a police report and provided breach notification to HHS, the media, and all affected individuals, as well as offering one year of free identity theft protection.  Following OCR's investigation, the CE's BA terminated the employee and re-trained its staff on its privacy and security policies, including not leaving laptops in unoccupied vehicles.  In addition, the CE reminded all contractors about the need to safeguard confidential information, and reviewed the BA's contractual obligations relating to safeguarding PHI.   The breach incident involved a BA and occurred prior to the September 23, 2013, compliance date.  OCR verified that the CE had a proper BA agreement in place that restricted the BA's use and disclosure of PHI and required the BA to safeguard all PHI. A binder containing the protected health information (PHI) of up to 1,272 individuals was stolen from a staff member's vehicle.  The PHI included names, telephone numbers, detailed treatment notes, and possibly social security numbers.  In response to the breach, the covered entity (CE) sanctioned the workforce member and developed a new policy requiring on-call staff members to submit any information created during their shifts to the main office instead of adding it to the binder.  Following OCR's investigation, the CE notified the local media about the breach.                                                                                                                                                                                                                                                                                                                                                                                                                                                                                                                                                                                                         
## highest: Two unencrypted desktop computers containing the electronic protected health information (ePHI) of 16,820 individuals were stolen from the covered entity (CE).  The ePHI included medical record numbers, dates of birth, admission /discharge dates, billing codes, and social security numbers.  Upon discovery of the breach, the CE filed a police report and provided breach notification to HHS, the media, and affected individuals.  It also provide substitute notification by posting on its website.  As a result of OCR's investigation, the CE replaced its building alarm and installed bars on the windows.  In addition, the CE directed its staff to save patient data only on a centralized network drive, moved all ePHI stored on desktop hard drives to centralized secured network servers, and encrypted all of its computers.   The CE also revised its policy and procedure on password management and provided training to all staff on its new policy.                                                                                                                                                                                                                 Two unencrypted laptop computers containing the electronic protected health information (ePHI) of 712 individuals were stolen from the covered entity's (CE) office.  The ePHI included names, dates of birth, social security numbers, diagnostic reports, and demographic information.  Upon discovery of the breach, the CE filed a police report to recover the stolen items.  As a result of OCR's investigation, the CE improved physical security by installing an exit alarm lock and surveillance camera, and implementing a policy and procedure requiring managers to monitor inappropriate use of the facility's rear exit.  The CE also inventoried its ePHI systems and adopted and implemented policies and procedures for workstation security, encryption, security awareness and training, electronic devices, and media controls.                                                                                                                                                                                                                                                                                                                                               Unencrypted clinical system backup tapes that contained the electronic protected health information (ePHI) of 1,700,000 individuals were stolen from the unlocked vehicle of an employee of the covered entity's (CE) business associate (BA).  The ePHI included names, medical record numbers, social security numbers, addresses, telephone numbers, health plan numbers, dates of birth, dates of admission, dates of treatment, dates of discharge, dates of death, mother's name, next of kin, clinical information related to diagnosis, treatment, prognosis, laboratory tests and results, and medications.  Upon discovery of the breach, the CE filed a police report to recover the stolen items and provided breach notification to HHS, the media, and affected individuals.  As a result of OCR's investigation, the CE terminated its BA agreement and installed encryption software on backup media.  The breach incident involved a BA and occurred prior to the September 23, 2013, compliance date.  OCR verified that the CE had a proper BA agreement in place that restricted the BA's use and disclosure of PHI and required the BA to safeguard all PHI.                  Upon request, a subcontractor (PHM Software Solutions) of the covered entity's (CE) business associate (BA), PHM Healthcare Solutions, modified a software application the CE was utilizing  which led to the disclosure of electronic protected health information (ePHI) of 5,000 individuals on the Internet.  The ePHI included names, gender, member identification numbers, dates of birth, and consent forms.  The CE provided breach notification to HHS, the media, and affected individuals and posted substitute notice on its website.  Upon discovery of the breach, the BA removed the software application and placed it offline.  As a result of OCR's investigation, the CE had its BA to conduct a risk analysis and create a risk management plan to address any vulnerabilities identified in the risk analysis.  The breach incident involved a BA and occurred prior to the September 23, 2013, compliance date.  OCR provided technical assistance to assist the CE understand its obligations under the Privacy and Security Rules regarding BA agreements.                                                                                                                Without permission from the covered entity (CE), an employee provided a list of patient's names to a local counseling center as the employee was leaving the CE to begin employment at the new counseling center in an attempt to coordinate care of the patients she was treating.  The list, containing the PHI of approximately 771 individuals, included names, dates of birth, addresses, phone numbers, names of the insurance carriers, and facility codes.  Following the disclosure, the CE provided breach notification to HHS, the media, and all individuals affected and sanctioned the former employee for violating its policies and procedures. The CE also changed its procedures for list management.  The CE sent a reminder to all of its health care providers regarding the handling of PHI and made plans to provide HIPAA compliance information in a quality assurance newsletter.                                                                                                                                                                                                                                                                                       
## --------------------------------------------------------------------------------
## breach_start 
##          n    missing   distinct       Info       Mean        Gmd        .05 
##       1055          0        732          1 2011-12-09      612.9 2009-10-31 
##        .10        .25        .50        .75        .90        .95 
## 2010-02-17 2010-11-08 2012-01-11 2013-03-07 2013-10-17 2014-01-09 
## 
## lowest : 1997-01-01 2002-05-06 2003-03-29 2004-04-21 2004-05-01
## highest: 2014-04-19 2014-05-13 2014-05-27 2014-05-30 2014-06-02
## --------------------------------------------------------------------------------
## breach_end 
##          n    missing   distinct       Info       Mean        Gmd        .05 
##        145        910        121          1 2012-10-28      279.6 2011-11-17 
##        .10        .25        .50        .75        .90        .95 
## 2011-12-19 2012-04-22 2012-10-29 2013-05-29 2013-08-15 2013-10-03 
## 
## lowest : 2007-06-14 2011-02-28 2011-08-05 2011-08-18 2011-09-20
## highest: 2013-10-15 2013-10-31 2013-11-06 2013-11-08 2013-11-30
## --------------------------------------------------------------------------------
\end{verbatim}

\hypertarget{resumen-del-dataset}{%
\subsection{1.5 Resumen del dataset}\label{resumen-del-dataset}}

\begin{verbatim}
##  Name_of_Covered_Entity     State     Business_Associate_Involved
##  Length:1055            CA     :113   Length:1055                
##  Class :character       TX     : 83   Class :character           
##  Mode  :character       FL     : 66   Mode  :character           
##                         NY     : 58                              
##                         IL     : 49                              
##                         IN     : 40                              
##                         (Other):646                              
##  Individuals_Affected                        Type_of_Breach
##  Min.   :    500      Theft                         :516   
##  1st Qu.:   1000      Unauthorized Access/Disclosure:150   
##  Median :   2300      Other                         : 91   
##  Mean   :  30262      Loss                          : 85   
##  3rd Qu.:   6941      Hacking/IT Incident           : 75   
##  Max.   :4900000      Improper Disposal             : 38   
##                       (Other)                       :100   
##                  Location_of_Breached_Information Date_Posted_or_Updated
##  Paper                           :227             Min.   :2014-01-23    
##  Laptop                          :217             1st Qu.:2014-01-23    
##  Other                           :116             Median :2014-01-23    
##  Desktop Computer                :113             Mean   :2014-02-23    
##  Network Server                  :107             3rd Qu.:2014-03-24    
##  Other Portable Electronic Device: 60             Max.   :2014-06-30    
##  (Other)                         :215                                   
##    Summary           breach_start          breach_end        
##  Length:1055        Min.   :1997-01-01   Min.   :2007-06-14  
##  Class :character   1st Qu.:2010-11-08   1st Qu.:2012-04-22  
##  Mode  :character   Median :2012-01-11   Median :2012-10-29  
##                     Mean   :2011-12-09   Mean   :2012-10-28  
##                     3rd Qu.:2013-03-07   3rd Qu.:2013-05-29  
##                     Max.   :2014-06-02   Max.   :2013-11-30  
##                                          NA's   :910
\end{verbatim}

\hypertarget{preguntas}{%
\subsection{2 Preguntas}\label{preguntas}}

\hypertarget{cuuxe1les-son-los-tipos-de-brechas-que-afectarona-muxe1s-personas}{%
\subsection{2.1 ¿Cuáles son los tipos de brechas que afectarona más
personas?}\label{cuuxe1les-son-los-tipos-de-brechas-que-afectarona-muxe1s-personas}}

\hypertarget{cual-es-el-tiempo-promedio-para-superar-la-brecha}{%
\subsection{2.2 ¿Cual es el tiempo promedio para superar la
brecha?}\label{cual-es-el-tiempo-promedio-para-superar-la-brecha}}

\hypertarget{quuxe9-tipo-de-almacenamiento-de-la-informaciuxf3n-tuvo-mas-vulnerabilidades}{%
\subsection{2.3 ¿Qué tipo de almacenamiento de la información tuvo mas
vulnerabilidades?}\label{quuxe9-tipo-de-almacenamiento-de-la-informaciuxf3n-tuvo-mas-vulnerabilidades}}

\hypertarget{cuuxe1les-son-los-estados-muxe1s-atacados}{%
\subsection{2.4 ¿Cuáles son los Estados más
atacados?}\label{cuuxe1les-son-los-estados-muxe1s-atacados}}

\hypertarget{cuuxe1ntas-empresas-afectaron-a-terceros-tras-un-ciberataque}{%
\subsection{2.5 ¿Cuántas empresas afectaron a terceros tras un
ciberataque?}\label{cuuxe1ntas-empresas-afectaron-a-terceros-tras-un-ciberataque}}

\begin{Shaded}
\begin{Highlighting}[]
\NormalTok{si }\OtherTok{\textless{}{-}}\NormalTok{ (}\FunctionTok{sum}\NormalTok{(}\SpecialCharTok{!}\FunctionTok{is.na}\NormalTok{(cyberb}\SpecialCharTok{$}\NormalTok{Business\_Associate\_Involved))}\SpecialCharTok{*}\DecValTok{100}\NormalTok{)}\SpecialCharTok{/}\FunctionTok{length}\NormalTok{(cyberb}\SpecialCharTok{$}\NormalTok{Business\_Associate\_Involved)}
\NormalTok{no }\OtherTok{\textless{}{-}}\NormalTok{ (}\FunctionTok{sum}\NormalTok{(}\FunctionTok{is.na}\NormalTok{(cyberb}\SpecialCharTok{$}\NormalTok{Business\_Associate\_Involved))}\SpecialCharTok{*}\DecValTok{100}\NormalTok{)}\SpecialCharTok{/}\FunctionTok{length}\NormalTok{(cyberb}\SpecialCharTok{$}\NormalTok{Business\_Associate\_Involved)}

\NormalTok{a5 }\OtherTok{\textless{}{-}} \FunctionTok{data.frame}\NormalTok{(}
  \AttributeTok{Respuesta=}\FunctionTok{c}\NormalTok{(}\StringTok{"SI"}\NormalTok{,}\StringTok{"NO"}\NormalTok{),}
  \AttributeTok{Porcentajte=}\FunctionTok{c}\NormalTok{(si,no)}
\NormalTok{)}
\end{Highlighting}
\end{Shaded}

\includegraphics{Trabajo-final_files/figure-latex/unnamed-chunk-6-1.pdf}

\hypertarget{las-10-empresas-que-tuvieron-la-mayor-cantidad-de-afectados}{%
\subsection{2.6 Las 10 empresas que tuvieron la mayor cantidad de
afectados}\label{las-10-empresas-que-tuvieron-la-mayor-cantidad-de-afectados}}

\begin{Shaded}
\begin{Highlighting}[]
\NormalTok{a6 }\OtherTok{\textless{}{-}}\NormalTok{ (}\FunctionTok{data.frame}\NormalTok{ (}\AttributeTok{ID =} \FunctionTok{paste}\NormalTok{(}\StringTok{"E"}\NormalTok{,}\FunctionTok{c}\NormalTok{(}\DecValTok{1}\SpecialCharTok{:}\FunctionTok{length}\NormalTok{(cyberb}\SpecialCharTok{$}\NormalTok{Name\_of\_Covered\_Entity)), }\AttributeTok{sep =} \StringTok{""}\NormalTok{),}
                   \AttributeTok{Entidad =}\NormalTok{ cyberb}\SpecialCharTok{$}\NormalTok{Name\_of\_Covered\_Entity,}
                   \AttributeTok{Individuos\_afectados =} \FunctionTok{trunc}\NormalTok{((cyberb}\SpecialCharTok{$}\NormalTok{Individuals\_Affected)}\SpecialCharTok{/}\DecValTok{1000}\NormalTok{))) }\SpecialCharTok{\%\textgreater{}\%} \FunctionTok{arrange}\NormalTok{(}\FunctionTok{desc}\NormalTok{(Individuos\_afectados)) }\SpecialCharTok{\%\textgreater{}\%} \FunctionTok{slice}\NormalTok{(}\DecValTok{1}\SpecialCharTok{:}\DecValTok{10}\NormalTok{)}
\end{Highlighting}
\end{Shaded}

\includegraphics{Trabajo-final_files/figure-latex/unnamed-chunk-8-1.pdf}

\begin{longtable}[]{@{}
  >{\raggedright\arraybackslash}p{(\columnwidth - 4\tabcolsep) * \real{0.03}}
  >{\raggedright\arraybackslash}p{(\columnwidth - 4\tabcolsep) * \real{0.82}}
  >{\raggedleft\arraybackslash}p{(\columnwidth - 4\tabcolsep) * \real{0.15}}@{}}
\caption{Top 5 empresas afectadas}\tabularnewline
\toprule
\begin{minipage}[b]{\linewidth}\raggedright
ID
\end{minipage} & \begin{minipage}[b]{\linewidth}\raggedright
Entidad
\end{minipage} & \begin{minipage}[b]{\linewidth}\raggedleft
Individuos\_afectados
\end{minipage} \\
\midrule
\endfirsthead
\toprule
\begin{minipage}[b]{\linewidth}\raggedright
ID
\end{minipage} & \begin{minipage}[b]{\linewidth}\raggedright
Entidad
\end{minipage} & \begin{minipage}[b]{\linewidth}\raggedleft
Individuos\_afectados
\end{minipage} \\
\midrule
\endhead
E410 & TRICARE Management Activity (TMA) & 4900 \\
E800 & Advocate Health and Hospitals Corporation, d/b/a Advocate Medical
Group & 4029 \\
E271 & Health Net, Inc. & 1900 \\
E243 & New York City Health \& Hospitals Corporation's North Bronx
Healthcare Network & 1700 \\
E93 & AvMed, Inc. & 1220 \\
E383 & The Nemours Foundation & 1055 \\
E186 & BlueCross BlueShield of Tennessee, Inc. & 1023 \\
E414 & Sutter Medical Foundation & 943 \\
E911 & Horizon Healthcare Services, Inc., doing business as Horizon Blue
Cross Blue Shield of New Jersey, and its affiliates & 839 \\
E118 & South Shore Hospital & 800 \\
\bottomrule
\end{longtable}

\end{document}
