% Options for packages loaded elsewhere
\PassOptionsToPackage{unicode}{hyperref}
\PassOptionsToPackage{hyphens}{url}
%
\documentclass[
]{article}
\title{Ernesto}
\author{}
\date{\vspace{-2.5em}}

\usepackage{amsmath,amssymb}
\usepackage{lmodern}
\usepackage{iftex}
\ifPDFTeX
  \usepackage[T1]{fontenc}
  \usepackage[utf8]{inputenc}
  \usepackage{textcomp} % provide euro and other symbols
\else % if luatex or xetex
  \usepackage{unicode-math}
  \defaultfontfeatures{Scale=MatchLowercase}
  \defaultfontfeatures[\rmfamily]{Ligatures=TeX,Scale=1}
\fi
% Use upquote if available, for straight quotes in verbatim environments
\IfFileExists{upquote.sty}{\usepackage{upquote}}{}
\IfFileExists{microtype.sty}{% use microtype if available
  \usepackage[]{microtype}
  \UseMicrotypeSet[protrusion]{basicmath} % disable protrusion for tt fonts
}{}
\makeatletter
\@ifundefined{KOMAClassName}{% if non-KOMA class
  \IfFileExists{parskip.sty}{%
    \usepackage{parskip}
  }{% else
    \setlength{\parindent}{0pt}
    \setlength{\parskip}{6pt plus 2pt minus 1pt}}
}{% if KOMA class
  \KOMAoptions{parskip=half}}
\makeatother
\usepackage{xcolor}
\IfFileExists{xurl.sty}{\usepackage{xurl}}{} % add URL line breaks if available
\IfFileExists{bookmark.sty}{\usepackage{bookmark}}{\usepackage{hyperref}}
\hypersetup{
  pdftitle={Ernesto},
  hidelinks,
  pdfcreator={LaTeX via pandoc}}
\urlstyle{same} % disable monospaced font for URLs
\usepackage[margin=1in]{geometry}
\usepackage{color}
\usepackage{fancyvrb}
\newcommand{\VerbBar}{|}
\newcommand{\VERB}{\Verb[commandchars=\\\{\}]}
\DefineVerbatimEnvironment{Highlighting}{Verbatim}{commandchars=\\\{\}}
% Add ',fontsize=\small' for more characters per line
\usepackage{framed}
\definecolor{shadecolor}{RGB}{248,248,248}
\newenvironment{Shaded}{\begin{snugshade}}{\end{snugshade}}
\newcommand{\AlertTok}[1]{\textcolor[rgb]{0.94,0.16,0.16}{#1}}
\newcommand{\AnnotationTok}[1]{\textcolor[rgb]{0.56,0.35,0.01}{\textbf{\textit{#1}}}}
\newcommand{\AttributeTok}[1]{\textcolor[rgb]{0.77,0.63,0.00}{#1}}
\newcommand{\BaseNTok}[1]{\textcolor[rgb]{0.00,0.00,0.81}{#1}}
\newcommand{\BuiltInTok}[1]{#1}
\newcommand{\CharTok}[1]{\textcolor[rgb]{0.31,0.60,0.02}{#1}}
\newcommand{\CommentTok}[1]{\textcolor[rgb]{0.56,0.35,0.01}{\textit{#1}}}
\newcommand{\CommentVarTok}[1]{\textcolor[rgb]{0.56,0.35,0.01}{\textbf{\textit{#1}}}}
\newcommand{\ConstantTok}[1]{\textcolor[rgb]{0.00,0.00,0.00}{#1}}
\newcommand{\ControlFlowTok}[1]{\textcolor[rgb]{0.13,0.29,0.53}{\textbf{#1}}}
\newcommand{\DataTypeTok}[1]{\textcolor[rgb]{0.13,0.29,0.53}{#1}}
\newcommand{\DecValTok}[1]{\textcolor[rgb]{0.00,0.00,0.81}{#1}}
\newcommand{\DocumentationTok}[1]{\textcolor[rgb]{0.56,0.35,0.01}{\textbf{\textit{#1}}}}
\newcommand{\ErrorTok}[1]{\textcolor[rgb]{0.64,0.00,0.00}{\textbf{#1}}}
\newcommand{\ExtensionTok}[1]{#1}
\newcommand{\FloatTok}[1]{\textcolor[rgb]{0.00,0.00,0.81}{#1}}
\newcommand{\FunctionTok}[1]{\textcolor[rgb]{0.00,0.00,0.00}{#1}}
\newcommand{\ImportTok}[1]{#1}
\newcommand{\InformationTok}[1]{\textcolor[rgb]{0.56,0.35,0.01}{\textbf{\textit{#1}}}}
\newcommand{\KeywordTok}[1]{\textcolor[rgb]{0.13,0.29,0.53}{\textbf{#1}}}
\newcommand{\NormalTok}[1]{#1}
\newcommand{\OperatorTok}[1]{\textcolor[rgb]{0.81,0.36,0.00}{\textbf{#1}}}
\newcommand{\OtherTok}[1]{\textcolor[rgb]{0.56,0.35,0.01}{#1}}
\newcommand{\PreprocessorTok}[1]{\textcolor[rgb]{0.56,0.35,0.01}{\textit{#1}}}
\newcommand{\RegionMarkerTok}[1]{#1}
\newcommand{\SpecialCharTok}[1]{\textcolor[rgb]{0.00,0.00,0.00}{#1}}
\newcommand{\SpecialStringTok}[1]{\textcolor[rgb]{0.31,0.60,0.02}{#1}}
\newcommand{\StringTok}[1]{\textcolor[rgb]{0.31,0.60,0.02}{#1}}
\newcommand{\VariableTok}[1]{\textcolor[rgb]{0.00,0.00,0.00}{#1}}
\newcommand{\VerbatimStringTok}[1]{\textcolor[rgb]{0.31,0.60,0.02}{#1}}
\newcommand{\WarningTok}[1]{\textcolor[rgb]{0.56,0.35,0.01}{\textbf{\textit{#1}}}}
\usepackage{graphicx}
\makeatletter
\def\maxwidth{\ifdim\Gin@nat@width>\linewidth\linewidth\else\Gin@nat@width\fi}
\def\maxheight{\ifdim\Gin@nat@height>\textheight\textheight\else\Gin@nat@height\fi}
\makeatother
% Scale images if necessary, so that they will not overflow the page
% margins by default, and it is still possible to overwrite the defaults
% using explicit options in \includegraphics[width, height, ...]{}
\setkeys{Gin}{width=\maxwidth,height=\maxheight,keepaspectratio}
% Set default figure placement to htbp
\makeatletter
\def\fps@figure{htbp}
\makeatother
\setlength{\emergencystretch}{3em} % prevent overfull lines
\providecommand{\tightlist}{%
  \setlength{\itemsep}{0pt}\setlength{\parskip}{0pt}}
\setcounter{secnumdepth}{-\maxdimen} % remove section numbering
\ifLuaTeX
  \usepackage{selnolig}  % disable illegal ligatures
\fi

\begin{document}
\maketitle

\begin{Shaded}
\begin{Highlighting}[]
\ControlFlowTok{if}\NormalTok{(}\SpecialCharTok{!}\FunctionTok{require}\NormalTok{(readr)) \{}\FunctionTok{install.packages}\NormalTok{(}\StringTok{"readr"}\NormalTok{)\}}
\end{Highlighting}
\end{Shaded}

\begin{verbatim}
## Loading required package: readr
\end{verbatim}

\begin{Shaded}
\begin{Highlighting}[]
\FunctionTok{library}\NormalTok{(}\StringTok{"readr"}\NormalTok{)}
\ControlFlowTok{if}\NormalTok{(}\SpecialCharTok{!}\FunctionTok{require}\NormalTok{(dplyr)) \{}\FunctionTok{install.packages}\NormalTok{(}\StringTok{\textquotesingle{}dplyr\textquotesingle{}}\NormalTok{)\}}
\end{Highlighting}
\end{Shaded}

\begin{verbatim}
## Loading required package: dplyr
\end{verbatim}

\begin{verbatim}
## 
## Attaching package: 'dplyr'
\end{verbatim}

\begin{verbatim}
## The following objects are masked from 'package:stats':
## 
##     filter, lag
\end{verbatim}

\begin{verbatim}
## The following objects are masked from 'package:base':
## 
##     intersect, setdiff, setequal, union
\end{verbatim}

\begin{Shaded}
\begin{Highlighting}[]
\FunctionTok{library}\NormalTok{(}\StringTok{"dplyr"}\NormalTok{)}
\ControlFlowTok{if}\NormalTok{(}\SpecialCharTok{!}\FunctionTok{require}\NormalTok{(ggplot2)) \{}\FunctionTok{install.packages}\NormalTok{(}\StringTok{\textquotesingle{}ggplot2\textquotesingle{}}\NormalTok{)\}}
\end{Highlighting}
\end{Shaded}

\begin{verbatim}
## Loading required package: ggplot2
\end{verbatim}

\begin{Shaded}
\begin{Highlighting}[]
\FunctionTok{library}\NormalTok{(}\StringTok{"ggplot2"}\NormalTok{)}
\ControlFlowTok{if}\NormalTok{(}\SpecialCharTok{!}\FunctionTok{require}\NormalTok{(tidyverse)) \{}\FunctionTok{install.packages}\NormalTok{(}\StringTok{\textquotesingle{}tidyverse\textquotesingle{}}\NormalTok{)\}}
\end{Highlighting}
\end{Shaded}

\begin{verbatim}
## Loading required package: tidyverse
\end{verbatim}

\begin{verbatim}
## -- Attaching packages --------------------------------------- tidyverse 1.3.1 --
\end{verbatim}

\begin{verbatim}
## v tibble  3.1.4     v stringr 1.4.0
## v tidyr   1.1.4     v forcats 0.5.1
## v purrr   0.3.4
\end{verbatim}

\begin{verbatim}
## -- Conflicts ------------------------------------------ tidyverse_conflicts() --
## x dplyr::filter() masks stats::filter()
## x dplyr::lag()    masks stats::lag()
\end{verbatim}

\begin{Shaded}
\begin{Highlighting}[]
\FunctionTok{library}\NormalTok{(}\StringTok{"tidyverse"}\NormalTok{)}
\ControlFlowTok{if}\NormalTok{(}\SpecialCharTok{!}\FunctionTok{require}\NormalTok{(ggthemes)) \{}\FunctionTok{install.packages}\NormalTok{(}\StringTok{\textquotesingle{}ggthemes\textquotesingle{}}\NormalTok{)\}}
\end{Highlighting}
\end{Shaded}

\begin{verbatim}
## Loading required package: ggthemes
\end{verbatim}

\begin{Shaded}
\begin{Highlighting}[]
\FunctionTok{library}\NormalTok{(}\StringTok{"ggthemes"}\NormalTok{)}
\ControlFlowTok{if}\NormalTok{(}\SpecialCharTok{!}\FunctionTok{require}\NormalTok{(ggthemes)) \{}\FunctionTok{install.packages}\NormalTok{(}\StringTok{\textquotesingle{}lubridate\textquotesingle{}}\NormalTok{)\}}
\FunctionTok{library}\NormalTok{(}\StringTok{"lubridate"}\NormalTok{)}
\end{Highlighting}
\end{Shaded}

\begin{verbatim}
## 
## Attaching package: 'lubridate'
\end{verbatim}

\begin{verbatim}
## The following objects are masked from 'package:base':
## 
##     date, intersect, setdiff, union
\end{verbatim}

\begin{Shaded}
\begin{Highlighting}[]
\CommentTok{\#==Pregunta 1====================================================}
\NormalTok{cyberb }\OtherTok{\textless{}{-}} \FunctionTok{read\_csv}\NormalTok{(}\StringTok{"Cyber Security Breaches.csv"}\NormalTok{)}
\end{Highlighting}
\end{Shaded}

\begin{verbatim}
## Rows: 1055 Columns: 10
\end{verbatim}

\begin{verbatim}
## -- Column specification --------------------------------------------------------
## Delimiter: ","
## chr (9): Name_of_Covered_Entity, State, Business_Associate_Involved, Type_of...
## dbl (1): Individuals_Affected
\end{verbatim}

\begin{verbatim}
## 
## i Use `spec()` to retrieve the full column specification for this data.
## i Specify the column types or set `show_col_types = FALSE` to quiet this message.
\end{verbatim}

\begin{Shaded}
\begin{Highlighting}[]
\NormalTok{cyberb}\SpecialCharTok{$}\NormalTok{Type\_of\_Breach }\OtherTok{\textless{}{-}} \FunctionTok{as.factor}\NormalTok{(cyberb}\SpecialCharTok{$}\NormalTok{Type\_of\_Breach)}
\NormalTok{cyberb}\SpecialCharTok{$}\NormalTok{Date\_Posted\_or\_Updated }\OtherTok{\textless{}{-}} \FunctionTok{as.Date}\NormalTok{(cyberb}\SpecialCharTok{$}\NormalTok{Date\_Posted\_or\_Updated,}\AttributeTok{format=}\StringTok{"\%d/\%m/\%Y"}\NormalTok{)}
\NormalTok{cyberb}\SpecialCharTok{$}\NormalTok{breach\_start }\OtherTok{\textless{}{-}} \FunctionTok{as.Date}\NormalTok{(cyberb}\SpecialCharTok{$}\NormalTok{breach\_start,}\AttributeTok{format=}\StringTok{"\%d/\%m/\%Y"}\NormalTok{)}
\NormalTok{cyberb}\SpecialCharTok{$}\NormalTok{breach\_end }\OtherTok{\textless{}{-}} \FunctionTok{as.Date}\NormalTok{(cyberb}\SpecialCharTok{$}\NormalTok{breach\_end,}\AttributeTok{format=}\StringTok{"\%d/\%m/\%Y"}\NormalTok{)}


\NormalTok{cb2 }\OtherTok{\textless{}{-}} \FunctionTok{data.frame}\NormalTok{(}\AttributeTok{Type\_of\_Breach=}\NormalTok{ cyberb}\SpecialCharTok{$}\NormalTok{Type\_of\_Breach,}\AttributeTok{Individuals\_Affected =}\NormalTok{ cyberb}\SpecialCharTok{$}\NormalTok{Individuals\_Affected)}

\NormalTok{by\_type }\OtherTok{\textless{}{-}}\NormalTok{ cb2 }\SpecialCharTok{\%\textgreater{}\%} \FunctionTok{group\_by}\NormalTok{(Type\_of\_Breach)}

\NormalTok{by\_type\_sum }\OtherTok{\textless{}{-}}\NormalTok{ by\_type }\SpecialCharTok{\%\textgreater{}\%} \FunctionTok{summarise}\NormalTok{(}\AttributeTok{Individuals\_Affected =} \FunctionTok{sum}\NormalTok{(Individuals\_Affected))}

\NormalTok{by\_type\_top5 }\OtherTok{\textless{}{-}}\NormalTok{  by\_type\_sum }\SpecialCharTok{\%\textgreater{}\%}
                  \FunctionTok{arrange}\NormalTok{(}\FunctionTok{desc}\NormalTok{(Individuals\_Affected))}\SpecialCharTok{\%\textgreater{}\%}
                  \FunctionTok{slice}\NormalTok{(}\DecValTok{1}\SpecialCharTok{:}\DecValTok{5}\NormalTok{)}

\FunctionTok{ggplot}\NormalTok{(}\AttributeTok{data =}\NormalTok{ by\_type\_top5,}
       \AttributeTok{mapping =} \FunctionTok{aes}\NormalTok{(Type\_of\_Breach, Individuals\_Affected))}\SpecialCharTok{+}
  \FunctionTok{geom\_col}\NormalTok{()}\SpecialCharTok{+}
  \FunctionTok{xlab}\NormalTok{(}\StringTok{"Tipos de brechas"}\NormalTok{)}\SpecialCharTok{+}
  \FunctionTok{ylab}\NormalTok{(}\StringTok{"Cantidad"}\NormalTok{)}\SpecialCharTok{+}
  \FunctionTok{ggtitle}\NormalTok{((}\StringTok{"El top 5 de tipos de brechas"}\NormalTok{))}\SpecialCharTok{+}
  \FunctionTok{theme\_stata}\NormalTok{()}
\end{Highlighting}
\end{Shaded}

\includegraphics{Ernesto_files/figure-latex/cars-1.pdf}

\begin{Shaded}
\begin{Highlighting}[]
\CommentTok{\#==Pregunta 2=========================================================}
\FunctionTok{mean}\NormalTok{((cyberb}\SpecialCharTok{$}\NormalTok{breach\_end }\SpecialCharTok{{-}}\NormalTok{ cyberb}\SpecialCharTok{$}\NormalTok{breach\_start), }\AttributeTok{na.rm =}\NormalTok{ T)}
\end{Highlighting}
\end{Shaded}

\begin{verbatim}
## Time difference of 243.6345 days
\end{verbatim}

\begin{Shaded}
\begin{Highlighting}[]
\CommentTok{\#===Pregunta 3==============================================================}

\NormalTok{cyberb}\SpecialCharTok{$}\NormalTok{Location\_of\_Breached\_Information }\OtherTok{\textless{}{-}} \FunctionTok{as.factor}\NormalTok{(cyberb}\SpecialCharTok{$}\NormalTok{Location\_of\_Breached\_Information)}

\NormalTok{dfc2 }\OtherTok{\textless{}{-}}\NormalTok{ cyberb }\SpecialCharTok{\%\textgreater{}\%}
    \FunctionTok{group\_by}\NormalTok{(Location\_of\_Breached\_Information) }\SpecialCharTok{\%\textgreater{}\%}
    \FunctionTok{summarise}\NormalTok{(}\AttributeTok{count =} \FunctionTok{n}\NormalTok{()) }\SpecialCharTok{\%\textgreater{}\%}
    \FunctionTok{top\_n}\NormalTok{(}\AttributeTok{n =} \DecValTok{5}\NormalTok{, }\AttributeTok{wt =}\NormalTok{ count)}

\FunctionTok{ggplot}\NormalTok{(}\AttributeTok{data =}\NormalTok{ dfc2,}
       \AttributeTok{mapping =} \FunctionTok{aes}\NormalTok{(}\AttributeTok{x =}\NormalTok{ Location\_of\_Breached\_Information,}\AttributeTok{y =}\NormalTok{ count))}\SpecialCharTok{+}
  \FunctionTok{geom\_col}\NormalTok{()}\SpecialCharTok{+}
  \FunctionTok{xlab}\NormalTok{(}\StringTok{"Medio de almacenamiento de información"}\NormalTok{)}\SpecialCharTok{+}
  \FunctionTok{ylab}\NormalTok{(}\StringTok{"Cantidad"}\NormalTok{)}\SpecialCharTok{+}
  \FunctionTok{ggtitle}\NormalTok{((}\StringTok{"El top 5 de tipos de brechas"}\NormalTok{))}\SpecialCharTok{+}
  \FunctionTok{theme\_stata}\NormalTok{()}
\end{Highlighting}
\end{Shaded}

\includegraphics{Ernesto_files/figure-latex/cars-2.pdf}

\begin{Shaded}
\begin{Highlighting}[]
\CommentTok{\#===Pregunta 4============================================================}
\NormalTok{cyberb}\SpecialCharTok{$}\NormalTok{State }\OtherTok{\textless{}{-}} \FunctionTok{as.factor}\NormalTok{(cyberb}\SpecialCharTok{$}\NormalTok{State)}

\NormalTok{dfc3 }\OtherTok{\textless{}{-}}\NormalTok{ cyberb }\SpecialCharTok{\%\textgreater{}\%}
    \FunctionTok{group\_by}\NormalTok{(State) }\SpecialCharTok{\%\textgreater{}\%}
    \FunctionTok{summarise}\NormalTok{(}\AttributeTok{count =} \FunctionTok{n}\NormalTok{()) }\SpecialCharTok{\%\textgreater{}\%}
    \FunctionTok{top\_n}\NormalTok{(}\AttributeTok{n =} \DecValTok{10}\NormalTok{, }\AttributeTok{wt =}\NormalTok{ count)}

\FunctionTok{ggplot}\NormalTok{(}\AttributeTok{data =}\NormalTok{ dfc3,}
       \AttributeTok{mapping =} \FunctionTok{aes}\NormalTok{(}\AttributeTok{x =}\NormalTok{ State,}\AttributeTok{y =}\NormalTok{ count))}\SpecialCharTok{+}
  \FunctionTok{geom\_col}\NormalTok{()}\SpecialCharTok{+}
  \FunctionTok{xlab}\NormalTok{(}\StringTok{"Medio de almacenamiento de información"}\NormalTok{)}\SpecialCharTok{+}
  \FunctionTok{ylab}\NormalTok{(}\StringTok{"Cantidad"}\NormalTok{)}\SpecialCharTok{+}
  \FunctionTok{ggtitle}\NormalTok{((}\StringTok{"El top 5 de estados atacados"}\NormalTok{))}\SpecialCharTok{+}
  \FunctionTok{theme\_stata}\NormalTok{()}
\end{Highlighting}
\end{Shaded}

\includegraphics{Ernesto_files/figure-latex/cars-3.pdf}

\begin{Shaded}
\begin{Highlighting}[]
\CommentTok{\#===Pregunta 5=========================================================}
\FunctionTok{sum}\NormalTok{(}\SpecialCharTok{!}\FunctionTok{is.na}\NormalTok{(cyberb}\SpecialCharTok{$}\NormalTok{Business\_Associate\_Involved))}
\end{Highlighting}
\end{Shaded}

\begin{verbatim}
## [1] 271
\end{verbatim}

\hypertarget{including-plots}{%
\subsection{Including Plots}\label{including-plots}}

You can also embed plots, for example:

\includegraphics{Ernesto_files/figure-latex/pressure-1.pdf}

Note that the \texttt{echo\ =\ FALSE} parameter was added to the code
chunk to prevent printing of the R code that generated the plot.

\end{document}
